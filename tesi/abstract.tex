\begin{center}
\LARGE{\textbf{Abstract}}
\end{center}
\vspace{3em}
In questo lavoro, dopo un'introduzione sul panorama contemporaneo, si è analizzato lo standard IEEE 802.21, illustrando la {\em timeline} del processo di standardizzazione, i suoi obbiettivi e l'architettura del sistema descritto. Dopo aver citato qualche esempio di possibile applicazione dello standard e lo stato attuale dell'arte, si è studiata una sua implementazione {\em cross-platform} chiamata ODTONE, descrivendone i vari componenti e le loro funzionalità, ma anche sottolineando le attuali mancanze per arrivare ad una implementazione completa sotto tutti i punti di vista. Successivamente si è studiata ed implementata una semplice applicazione che potesse sfruttare in modo costruttivo le informazioni accessibili tramite lo standard per poter scegliere su quale interfacce instradare i pacchetti in due versioni, unidirezionale e bidirezionale.