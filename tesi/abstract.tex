\begin{center}
\LARGE{\textbf{Abstract}}
\end{center}
\vspace{3em}
In questo lavoro, dopo un'introduzione sul panorama contemporaneo, si è analizzato lo standard IEEE 802.21, illustrandone i motivi che hanno portato al suo sviluppo, la {\em timeline} del processo di standardizzazione, gli obbiettivi del {\em working group}, l'architettura del sistema specificato e le sue funzionalità, con particolare riguardo all'utilità in applicazioni reali, al fine di darne un giudizio completo sulla sua effettiva efficacia. Dopo aver citato qualche esempio di possibile applicazione dello standard e descritto lo stato attuale dell'arte, si è studiata una sua implementazione {\em cross-platform} chiamata ODTONE, descrivendone i vari componenti e le loro funzionalità, ma anche sottolineando le attuali mancanze per arrivare ad una implementazione completa sotto tutti i punti di vista. Successivamente si è studiata ed implementata un'applicazione, {\em MIH-proxy}, che potesse sfruttare in modo costruttivo i servizi specificati dallo standard per creare un proxy che potesse scegliere su quale interfaccia instradare i pacchetti a seconda dello stato attuale di tutti i collegamenti, realizzato in versione unidirezionale e bidirezionale. In particolare questa applicazione è in grado di restare in ascolto di cambiamenti di stato delle interfacce di rete, e.g. quando viene stabilita una connessione oppure cade, e, di conseguenza, stabilire di volta in volta quali collegamenti utilizzare per inviare dati. Nella versione bidirezionale è anche possibile far comunicare tra loro applicazioni che normalmente utilizzerebbero il protocollo di trasporto TCP attraverso un ulteriore componente, {\em phoxy}, che si preoccupa di convertire, in modo trasparente, un flusso TCP in datagrammi UDP eventualmente cifrati. Sarà quindi possibile creare un collegamento criptato ad alta affidabilità tra le applicazioni che possa sfruttare tutte le interfacce disponibili, sia per inviare, sia per ricevere.