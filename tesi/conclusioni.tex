\chapter{Conclusioni}

\section{Dopo il lavoro svolto}
Durante la realizzazione di questo lavoro è stato possibile apprezzare lo standard IEEE 802.21 sia a livello teorico, sia a livello pratico. Lo standard in sé specifica funzionalità molto utili per poter prendere decisioni di {\em handover} e, anche solo testando un sottoinsieme delle funzionalità teoricamente offerte attraverso l'implementazione open-source ODTONE, si è rivelato efficace per gestire a livello utente dei meccanismi di {\em handover} per un singolo flusso di dati e creare così un proxy ad alta affidabilità, sfruttando tutti i collegamenti disponibili. Per quanto riguarda ODTONE, la principale limitazione riscontrata sta nel fatto che i Link-SAPs forniti da ODTONE sono solo per le tecnologie 802.3 e 802.21, quindi non è possibile utilizzare nessun protocollo della famiglia 3GPP per eseguire dei test.

\section{Sviluppi futuri}
Bisognerebbe nel prossimo futuro completare ed ampliare il progetto ODTONE, in particolar modo riguardo i SAPs disponibili attualmente e gli eventi generabili, in modo da poter testare più fedelmente gli scenari del mondo odierno. Se fosse possibile testare anche una sola tecnologia non-IEEE 802 si potrebbe, ad esempio, fare qualche esperimento con dispositivi mobili reali, gestendone le decisioni di {\em handover}, come uno {\em smartphone} di nuova generazione con un sistema Android. Per quanto riguarda lo standard è necessario ora insistere sul suo effettivo {\em deployment} e completare il lavoro integrandolo con tutti i meccanismi necessari per renderlo efficace, a partire da soluzioni di mobilità per quanto riguarda l'indirizzo IP e specifiche per l'effettiva esecuzione dell'{\em handover} tra una tecnologia e l'altra, che hanno bisogno di meccanismi per effettuare passaggi tra stessa tecnologia ({\em horizontal handover}) e tra tecnologie diverse ({\em vertical handover}).